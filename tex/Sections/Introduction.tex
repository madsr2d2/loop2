\section[Introduction]{Introduction\footnote{Generative AI was used as a writing aid throughout the development of this document.}}

Egholm A/S is a recognized leader in the development and manufacture of utility machines for municipalities and the outdoor maintenance sector. As the industry moves decisively toward sustainability~\cite{eu_green_deal}, Egholm has begun electrifying its product line. This transition presents a complex challenge: electric machines must meet the rigorous 8-hour workday demands of municipal operators while managing the inherent limitations of battery capacity.

Currently, Egholm's electric models face a runtime barrier, often falling short of the full workday requirement under standard conditions. While technical solutions such as larger batteries exist, they drive up costs and weight, potentially pricing the machines out of the competitive market.

Based on insights gathered during our visit to Egholm's production facility and correspondence with their team , we have identified that this challenge is not purely hardware-related. A significant portion of energy consumption is dictated by operator behavior [\ref{app:visitnotes}, \ref{app:mail}].This initial data suggests that operators, accustomed to diesel engines, often default to maximum power settings regardless of the task at hand---vacuuming dry leaves with the same force used for wet debris, for example.

This report explores the potential of the \textbf{Energy-Aware Assistance System (EAAS)}, a solution designed to bridge the gap between technical capability and human usage. By leveraging behavioral design and real-time feedback, we aim to demonstrate how Egholm A/S can extend machine runtime and reduce costs without expensive hardware upgrades.

The following sections will detail our analysis of this problem, the development of our EAAS prototype, and the business case for implementing such a system in the next generation of Egholm A/S utility machines.
