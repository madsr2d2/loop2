\section{Conclusion}

The transition to electric utility machines represents both a technical challenge and a significant market opportunity for Egholm A/S. Our investigation confirmed that while battery capacity is a limiting factor, operator behavior plays a decisive role in real-world energy consumption. The current lack of feedback leads to a default ``max-power'' driving style that unnecessarily drains batteries and shortens runtime.

To address this, we proposed the \textbf{Energy-Aware Assistance System (EAAS)}. This concept moves beyond purely mechanical optimization to include the human element of the system. By providing operators with simple, real-time feedback on their energy efficiency, Egholm can achieve:

\begin{enumerate}
    \item \textbf{Extended Runtime:} Reducing unnecessary power usage directly translates to longer operating hours, addressing the primary concern of municipal customers.
    \item \textbf{Cost Efficiency:} Optimizing usage reduces the immediate need for larger, more expensive battery packs, preserving Egholm's competitive pricing structure.
    \item \textbf{Future-Proofing:} The implementation of EAAS serves as a stepping stone toward full telematics and AI-driven predictive maintenance, positioning Egholm as a leader in smart, sustainable municipal maintenance.
\end{enumerate}

We recommend that Egholm prioritizes the development of this behavioral interface alongside their hardware electrification. The ``Hard Nut'' of energy efficiency is not just about storing more power; it is about using the available power smarter. EAAS provides the tool to make that smart usage a daily habit for every operator.