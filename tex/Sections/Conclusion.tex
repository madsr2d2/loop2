\section{Conclusion}
Loop 1 has been an important opportunity to explore a complex real-world challenge and transform it into a concrete, actionable proposal. Starting from the broad ambition of Roskilde Municipality to foster energy communities, we identified the “hard nut” as the lack of decision-ready information for housing associations considering solar energy. Our prototype demonstrated that by leveraging publicly available datasets, it is possible to generate transparent, address-specific briefs that can support both citizens and boards in making informed decisions.

The economic analysis confirmed that rooftop solar is becoming increasingly attractive in Denmark, not only for individual households but also for municipalities striving toward sustainable energy targets. However, our work also revealed that technology alone is insufficient; adoption depends on clarity, trust, and the ability to communicate complex information in a simple and actionable way.

Although Loop 2 will focus on an entirely different project, the learnings from this loop remain highly relevant. We have gained experience in framing challenges around stakeholder needs, testing ideas through rapid prototyping, and ensuring transparency in data and assumptions. These methods and insights will help us approach future innovation projects with sharper focus, stronger stakeholder alignment, and a better understanding of how to translate technical feasibility into real-world adoption.

In this sense, the greatest outcome of Loop 1 is not only the prototype itself but also the transferable skills and perspectives it has given us. These will serve as a foundation for tackling new challenges in the next loop and beyond. 