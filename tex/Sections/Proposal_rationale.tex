\section{Proposal}\label{sec:proposal}

\subsection{Problem statement}
Boards of andelsboligforeninger / boligforeninger in Roskilde struggle to take the first step toward forming energifællesskaber (energy communities) and investing in rooftop photovoltaics (PV). Today, information is fragmented across public registers and platforms, and boards often lack the time and analytical skills to combine these sources into credible, comparable pre-feasibility economics.

The result is slow or abandoned initiatives, “ping-pong” with installers for basic numbers, and GA (General Assembly) agendas without decision-ready material. A neutral, address-specific, source-linked brief with transparent assumptions and confidence bands is needed to move from idea $\rightarrow$ pilot/tender. Residents also need a way to self-start by producing the same decision-ready brief for the board. At municipal scale, a portfolio view is needed to surface high-potential roofs (especially where roof renovation is due) so outreach and pilots can be prioritised.\cite{HFE,KEF, erfaringer,webpage1}

\subsection{Stakeholders}
\begin{itemize}
  \item \textbf{Housing association boards} — make GA decisions on capital expenditure (CAPEX). Pain: lack of time, fragmented data, inconsistent assumptions. Need: concise, neutral pre-feasibility briefs (address-specific economics, assumptions, confidence bands) they can include in GA packets.
  \item \textbf{Residents (including ildsjæle)} — initiate ideas and seek board attention. Pain: no credible, shareable numbers. Need: a self-service path to generate a proposal-ready brief and email/share it with the board.
  \item \textbf{Property administrators} — support boards with numbers and compliance. Pain: manual data collection, unclear data lineage. Need: source-linked briefs and comma-separated values (CSV) exports they can archive and reuse across sites.
  \item \textbf{Installers/EPCs (Engineering, Procurement \& Construction)} — provide quotes and designs. Pain: incomplete inputs and repeated clarification. Need: a standardised, address-specific brief that speeds up quoting and reduces back-and-forth.
  \item \textbf{Municipality (energy/climate and permitting)} — pursues climate targets and pilots. Pain: hard to identify and prioritise candidate roofs. Need: portfolio screening (including likely roof-renovation timing) and a transparent audit trail for engagement.
  \item \textbf{Distribution System Operator (DSO)} — grid connection. Pain: early requests without context. Need: basic site parameters up front so interconnection discussions start on solid ground.
  \item \textbf{Public data providers} — publish authoritative datasets. Pain: misuse/misinterpretation. Need: clear citation and retrieval dates in briefs to maintain trust.
  \item \textbf{Funders/banks} — support financing when projects mature. Pain: non-comparable estimates. Need: consistent, source-linked pre-feasibility numbers that can be diligence-checked later.
\end{itemize}

\subsection{Vision}
A self-service pre-feasibility portal that converts any Roskilde address into a decision-ready brief in minutes. The brief is address-specific and source-linked, with transparent assumptions and confidence bands, so boards can move from idea to pilot/tender and residents can credibly approach the board. 

In addition to single-site use, the portal provides a portfolio screening mode to surface high-potential roofs across the municipal building stock and housing associations. Using public registers (e.g., BBR and related datasets), it can flag buildings likely nearing roof renovation (based on building age, roof material, and time since last recorded works), making them especially suitable candidates with lower incremental CAPEX when integrating PV (shared scaffolding, mounting integration). Each brief exports as a two-page PDF for GA packets and installer outreach, with a built-in share/email action for residents and comma-separated values (CSV) export for administrators

\subsection{Evidence from desk research}

We conducted desk research using publicly available sources to answer exactly what the Proposal needs: can we produce a neutral, address-specific pre-feasibility brief and a lightweight portfolio screen from public data only?

\subsubsection{Scope \& limitations}
No interviews or field studies. The brief is indicative pre-feasibility (not an engineering design). Findings are triangulated across two or more sources where possible. Building-level smart-meter data is not available via public APIs; for MVP we use representative/segment-level load shapes with sensitivity bands and can swap in bills/meter data during quote/pilot.

\subsubsection{Key findings}
\begin{itemize}
  \item Public data is sufficient to auto-generate a decision-ready, address-specific brief. Key sources include:
    \begin{itemize}
      \item DAWA — \url{https://dawadocs.dataforsyningen.dk/} (authoritative address lookup/geocoding; stable IDs for joins)
      \item Datafordeler (BBR, DAR, Matriklen) — \url{https://datafordeler.dk/} (building, address, and cadastral [property/parcel] registers)
      \item sologvindinfo.dk — \url{https://www.sologvindinfo.dk/} (per-roof solar potential for indicative PV sizing and yield)
      \item Energi Data Service — \url{https://energidataservice.dk/} (day-ahead prices DK1/DK2, CO$_2$ intensity, aggregated consumption/production, network/transmission tariff datasets)
    \end{itemize}
  \item Representative load profiles informed by Energi Data Service context are adequate for MVP; we present ranges and explicitly state assumptions.
  \item Many housing associations operate behind a shared main meter, supporting the early focus on associations and simplifying self-consumption accounting in initial phases.
  \item A neutral, source-linked presentation with assumptions and confidence bands differentiates the brief from installer calculators and reduces back-and-forth during quoting.
  \item Portfolio screening is feasible; BBR attributes (build/alteration year, roof material) can be used as proxies to flag likely roof-renovation timing. These are signals, not confirmations, and require board verification.
  \item For MVP, sologvindinfo.dk substitutes for custom PVGIS modelling.
\end{itemize}

\subsection{Proposal outline}\label{proposalOutline}
Build a web-based “Solar Readiness Brief” for housing association boards and residents. The user enters an address, the tool aggregates public data, and it estimates PV capacity (kWp), annual yield (MWh/yr), self-consumption, export, Capital Expenditure (CAPEX), Operating Expenditure (OPEX), and simple payback. It then generates a two-page PDF for decision-making and installer outreach. 

Each brief includes an assumptions panel, data lineage (source and retrieval date), and confidence bands to make uncertainty explicit. A built-in call to action lets residents email the brief to their board, while administrators can export a comma-separated values (CSV) file to prioritise sites and coordinate installer quotes. 

Beyond single-site reports, the portal supports portfolio screening of Roskilde’s municipal building stock to identify high-potential candidates for energy communities. Using public registers (e.g., BBR and related datasets), the tool can also flag buildings that are likely approaching roof renovation based on building age, roof material, and time since last recorded major works. These sites are especially suitable candidates because boards are already focused on the roof decision and the incremental CAPEX of integrating PV (shared scaffolding, mounting integration) is typically lower. Outreach to selected boards can be handled via GDPR-compliant email using the auto-generated brief and summary metrics, enabling targeted engagement without manual compilation.

\subsubsection{Needs it fulfills for stakeholders}
\begin{itemize}
    \item \textbf{Boards:} Faster, trustworthy pre-feasibility for General Assembly (GA) decisions. The brief provides plain-language results, an assumptions panel, and data lineage (source and retrieval date), reducing back-and-forth with administrators and installers. Example KPI: time-to-brief $\le$ 5 minutes; GA-readiness score from user testing $\ge$ target.

    \item \textbf{Residents:} A concrete path to initiate projects: enter an address $\rightarrow$ generate an address-specific brief $\rightarrow$ use the built-in email call-to-action to send it to the board. The shareable PDF helps place PV on the GA agenda without technical jargon. Example KPI: share-to-board rate; \% of resident-generated briefs that become GA agenda items within one cycle.

    \item \textbf{Roskilde Municipality:} A scalable pipeline of qualified projects. Portfolio screening highlights high-potential rooftops across the municipal building stock; comma-separated values (CSV) exports support prioritisation and coordination of installer quotes. The process is GDPR-compliant and creates a transparent audit trail through explicit data lineage. Example KPI: number of qualified sites per screening run; proportion progressing to request for quotation (RFQ)/pilot.

    \item \textbf{Regulatory \& ESG:} Support for energy-community formation and local generation while preserving trust. Transparency (assumptions, confidence bands, sources) and privacy practices (no personally identifiable information beyond address) align with GDPR and relevant electrical/building codes and DSO interconnection guidance. Example KPI: compliance checklist completion; \% of briefs with complete source citations.
\end{itemize}


\subsubsection{Decision rationale}
This tool is not limited to housing associations. It can screen businesses, public buildings, and detached homes as well. For this Minimum Viable Product (MVP) we focus on housing associations where activation is most straightforward. Estimates are produced with segment-specific models (load shapes, tariffs, metering/settlement rules), so results will differ by building type.

\paragraph{Why housing associations first:}
\begin{itemize}
  \item Information and capacity gap — boards often lack the time, focus, and analytical skills to combine public datasets into credible economic estimates. The portal’s concise, source-linked feasibility briefs close this gap and make GA discussions actionable.
  \item Established governance and decision authority — boards and committees are already in place with clear decision rules; General Assemblies (GAs) can authorise capital expenditure (CAPEX) in a single vote. A shared property and cost base aligns member priorities and simplifies orchestration of an energy community.
  \item Favourable roof-to-load match — large, contiguous roofs and aggregated common-area loads raise on-site self-consumption and improve payback; block-scale geometry also reduces shading-model uncertainty.
  \item Streamlined metering and settlement — many associations sit behind a shared main meter, enabling straightforward self-consumption accounting in early phases and avoiding per-apartment meter upgrades in the MVP.
\end{itemize}

\paragraph{Other segments}
Energy communities that share energy across the public distribution grid face common constraints: network tariffs and fees on exchanged energy; metering and settlement requirements; and the need for an appropriate legal/governance setup between participants. Moreover, auto-generating a reliable feasibility brief is harder for heterogeneous, geographically separated member groups, and portfolio screening/optimisation across multiple addresses is computationally intensive.

\paragraph{Businesses:}
\begin{itemize}
    \item \textbf{Pros:} Complementary, offsetting load profiles (e.g., offices by day, hospitality in the evening, cold storage overnight) can lift portfolio self-consumption (a key KPI); larger rooftops and predictable operating hours simplify sizing.
    \item \textbf{Cons:} Limited access to site-level interval data; roof ownership/lease splits; tenancy churn shifts loads; approval chains across facilities and finance slow decisions.

\end{itemize}

\paragraph{Public buildings:}
\begin{itemize}
    \item \textbf{Pros:} Portfolio scale and relatively stable loads; alignment with public climate targets; ability to bundle sites.
    \item \textbf{Cons:} Procurement law; annual budget cycles; heritage/listed-building constraints; cross-department coordination that extends timelines.
\end{itemize}

\paragraph{Detached homes:}
\begin{itemize}
    \item \textbf{Pros:} Potential for street- or development-based cohorts with similar roof types; large aggregate rooftop area; strong community appeal.
    \item \textbf{Cons:} High variance in consumption and roof condition; aesthetic and neighbourhood constraints; one-to-one outreach economics; accuracy depends on access to smart-meter data.
\end{itemize}

\subsection{Validation \& MVP targets}

\paragraph{Planned methods}
\begin{itemize}
  \item Benchmark outputs against trusted references: compare portal yield and sizing against sologvindinfo.dk and at least one established PV tool for representative rooftops; spot-check geometry via DSM/DTM and orthophotos.
  \item Run sensitivity analyses on key drivers: tariffs (Energi Data Service), representative load profiles, PV CAPEX/OPEX ranges, and export/self-consumption assumptions; report bands in the brief.
  \item Sanity-check economics against installer quotations when available and verify interconnection assumptions with DSO guidance.
  \item Lightweight usability tests with target users (board members, residents/ildsjæle, administrators): measure time-to-brief, comprehension of assumptions/confidence bands, and share-to-board flow.
\end{itemize}

\paragraph{MVP targets}
\begin{itemize}
  \item Usability: System Usability Scale (SUS) $\ge 70$. Time-to-brief $\le 5$ minutes. Task completion in tests $\ge 90\%$.
  \item Technical accuracy: annual yield error within $\pm 15\%$ vs reference (sologvindinfo.dk or equivalent). Economics within $\pm 20\%$ of installer Total Cost of Ownership (TCO) where quotes exist.
  \item Adoption: $\ge 35\%$ of boards presented with a brief request a pilot/quote. $\ge 20\%$ of resident-generated briefs appear as GA agenda items within one cycle.
  \item Data quality \& compliance: $\ge 95\%$ of briefs include source citations and retrieval dates. No personally identifiable information beyond address. Assumptions and confidence bands shown on every brief.
\end{itemize}

