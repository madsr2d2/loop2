\section{Business and Economic Impact}

During our visit to Egholm’s offices and production facilities, we gained a clear understanding of the economic constraints and opportunities involved in developing the next generation of electric utility machines. A central insight was that the battery pack accounts for approximately \textbf{25\% of the total production cost}~\cite{bnef2023}. Consequently, any innovation that extends runtime without increasing battery size has a direct, positive impact on the machine's unit economics and market competitiveness.

\subsection{Cost Drivers and Development}
Egholm currently invests heavily in physical testing to validate battery performance across diverse conditions. We learned that prototypes are driven to locations like Switzerland to test performance in extreme cold—a necessary but expensive and logistically complex process. While these tests ensure reliability, they represent a significant fixed cost in the development cycle.

We identified a strong business potential in shifting from purely physical testing to a data-driven approach~\cite{osterwalder2010business}. By combining Egholm’s initial baseline measurements with the proposed EAAS framework, the company can reduce its reliance on expensive field trips.

\subsection{Operational Optimization via Data}
Before full telematics can deliver precise automated recommendations, Egholm needs to establish a "reference map" of energy usage. The EAAS system facilitates this by gathering data on:
\begin{itemize}
	\item \textbf{Task-Specific Power Needs:} Differentiating requirements for dry vs. wet grass.
	\item \textbf{Load Variations:} Measuring how surface conditions (e.g., gravel, asphalt) affect resistance.
	\item \textbf{Environmental Factors:} Quantifying the impact of moisture and temperature on consumption.
\end{itemize}

Integrating this data flow offers two major economic advantages:
\begin{enumerate}
	\item \textbf{Scalability:} Once the sensor framework is deployed, Egholm collects data from the entire fleet at near-zero marginal cost, turning optimization into a continuous, scalable process.
	\item \textbf{Reduced Testing Costs:} Real-world customer data can supplement or partially replace expensive international stress tests, providing faster feedback loops at a lower cost.
\end{enumerate}

\subsection{Strategic Competitive Advantage}
For municipal customers, the value proposition is straightforward: intelligent power management translates to longer runtime per charge and a lower Total Cost of Ownership (TCO). Since the battery is a dominant cost factor, documenting that the machine utilizes its capacity with maximum efficiency becomes a powerful sales argument.

Overall, the EAAS approach supports a robust business case, enabling Egholm to:
\begin{itemize}
	\item \textbf{Reduce Development Costs:} By streamlining the testing and validation phase.
	\item \textbf{Accelerate Innovation:} Shortening the feedback loop for product improvements.
	\item \textbf{Strengthen ESG Profile:} Providing documented energy savings for sustainability reporting.
	\item \textbf{Differentiate Market Position:} Moving beyond hardware specs to offer "intelligent" machinery.
\end{itemize}

In conclusion, energy optimization is not merely a technical feature but a strategic business tool that addresses Egholm's most significant cost driver while delivering tangible value to the customer.
