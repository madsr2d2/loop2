\section{Business and Economic Perspective}
For private households in Denmark, the interest in solar panels has increased significantly in recent years, partly because the price of the solar panel system itself has fallen, and partly because electricity prices have been rising. An average estimate of the total costs of electricity for a private household today typically lies in the range of DKK 2.50–2.90 per kWh, including taxes, VAT, and grid tariffs. According to the Electricity Price Statistics from the Danish Utility Regulator, as reported by Elberegner.dk, the overall cost for a household consuming 4,000 kWh per year in the fourth quarter of 2024 was about DKK 2.90/kWh.\cite{bjorni1}

This price is consistent with Statistics Denmark, which shows that the average electricity price for households in the first half of 2025 was around DKK 2.6/kWh including taxes and fees.\cite{bjorni2} This price is the reason why many private households have been or may become motivated to reduce electricity consumption through self-consumption via solar panels on the roof.

The economic gain from a solar panel system naturally follows from how much of the electricity is used by the household itself, but also what can be earned from the surplus electricity. The part of the electricity that is not used in the household and is transferred to the grid is calculated at a fixed rate or based on market prices. Andel Energi offers a production agreement where surplus production is purchased at the spot price level, and in their calculations, DKK 0.55/kWh is used as a typical reference price.\cite{bjorni3}

In addition, the grid tariffs also play a role in the overall economy. According to Energinet’s forecast for electricity tariffs in the 2024–2026 period, tariffs are expected to rise as more energy will be used in society. More electric cars, heat pumps, and decentralized production mean that the electricity grid must handle greater loads and requires investments in infrastructure.\cite{bjorni4}

Energinet has also stated that in the near future, they will harmonize tariff payments and introduce so-called real-time tariffing for producers of their own electricity, such as private solar panel owners. This means that one will increasingly be billed based on the actual times when electricity is produced and consumed, instead of an average over the whole day. The goal is to create a fairer and more efficient system that promotes flexible electricity use and reduces peak load periods.\cite{bjorni5}

In summary, these economic conditions mean that solar panels for many households represent both a stable and manageable investment. A typical detached house with a system of about 4–6 kW can on average produce 4,000–6,000 kWh per year, of which about two-thirds can normally be used directly in the household. With the current electricity prices and settlement terms, this implies an annual saving of about DKK 5,000–9,000, of course depending on consumption pattern, roof size, and roof angle.
4–6 kW:\cite{bjorni6,bjorni7}

In a broader perspective, the spread of solar panels contributes not only to private economic benefits but also to a more sustainable local energy supply. For every citizen who produces their own electricity, the pressure on the electricity grid during peak load periods is reduced, and municipalities such as Roskilde can achieve a greener energy mix without becoming dependent on large, central plants. At the same time, it strengthens local business development, as more installers, electricians, and energy consultants get new tasks.

Solar panels are not only an economic advantage in some cases but a shared project for the coming years. It is an invitation to those citizens who want to help make a difference and who might only need the first argument to bring the idea to the board meeting or community meeting.
