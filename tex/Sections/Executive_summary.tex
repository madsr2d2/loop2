\section{Executive Summary}

Egholm A/S stands at a pivotal transition point as it electrifies its portfolio of utility machines. While the shift to electric drivetrains offers significant environmental and noise-reduction benefits, it introduces a critical challenge: battery runtime. Current electric models often face a runtime barrier, falling short of the 8-hour workday expected by municipal customers. Our analysis reveals that this limitation is not solely technical but largely behavioral; operators often default to maximum power settings regardless of the task's actual requirements, leading to unnecessary energy waste.

This report proposes the \textbf{Energy-Aware Assistance System (EAAS)}, a behavioral design solution integrated into the tractor's interface. Instead of forcing restrictions or requiring complex manual adjustments, EAAS uses real-time sensor data to provide intuitive, non-intrusive feedback---such as a simple color-coded display---that nudges operators toward more efficient behavior.

\textbf{Key findings and value proposition:}
\begin{itemize}
	\item \textbf{The ``Hard Nut'':} The core problem is the disconnect between operator habits (formed on diesel machines) and the energy-sensitivity of electric systems.
	\item \textbf{The Solution:} EAAS acts as a digital co-pilot, translating complex load data into simple actionable cues (e.g., ``Green'' for optimal, ``Red'' for wasteful).
	\item \textbf{Business Impact:} For Egholm, this system reduces the pressure to install larger, cost-prohibitive batteries (which account for ~25\% of production costs). For customers, it extends daily runtime and lowers the total cost of ownership.
	\item \textbf{Strategic Scalability:} The system lays the foundation for future telematics and AI-driven optimization, transforming energy efficiency from a static hardware feature into a dynamic, data-driven competitive advantage.
\end{itemize}

By bridging the gap between machine capability and human behavior, EAAS ensures that Egholm's electric tractors can deliver both the performance operators expect and the sustainability municipalities demand.
