\section{Mail from Eggholm A/S}
\label{app:mail}

\textbf{From:} Eggholm A/S Representative \\
\textbf{To:} Martin \\
\textbf{Subject:} Response to Questions about Case Project \\
\textbf{Date:} October 2023

\begin{quote}
	Hej Martin,

	Jeg skriver på vegne af Gruppe 58, der har fået jeres case. Vi glæder os super meget til at arbejde med jeres firma. Vi har nogle praktiske spørgsmål som vi håber du kan hjælpe med at svare på så vi kan komme forberedt på onsdag 29/10.

	\begin{enumerate}
		\item Indsamler I allerede nogle former for data som vi kan få adgang til? Bruger I GPS så man ikke kører over det samme sted 2 gange? Har I noget data på hvornår jeres maskiner kører optimalt $\to$ ikke optimalt?

		      I dag anvender Egholm ikke 'GPS' overvågning på maskiner i drift. Dog har vi i forbindelse med det seneste projekt arbejdet med en leverandør omkring en fremtidig telematik løsning.

		\item I spørger om det er muligt at indarbejde en form for motivation for brugeren af maskinerne til at bruge dem på en måde, hvor de bruger mindst mulig energi.

		      Vi ved at brugere anvender maskinerne forskelligt, f.eks. max. sugekapacitet på opsamlingstanken, også når det ikke er nødvendigt.

		      Er det fordi I allerede ved hvilke områder der kan forbedres?

		\item Har I haft prøvet at løse disse problemer før, hvis ja, har i haft nogle succeser? Har I mødt nogle "dead ends"?

		      Vi har bl.a. indført en række forbedringer i produkterne (f.eks. SW) som er uafhængige af brugerens adfærd.

		      Udgangspunktet for brugeren af maskinen er at opgaves løses bedst mulig og mindre fokus på energiforbrug. Det er her vi ønsker ideer der kan være med til at ændre brugerens adfærd.

		\item Hvad er jeres motivation for at forbedre kundernes brug af maskinerne?

		      Forbedret brug af maskiner skal komme både brugere (f.eks. mindre energiforbrug$\to$længere driftstid), kunder (f.eks. reduceret støj, støv mm.) og ikke mindst en lavere miljøbelastning.
	\end{enumerate}
\end{quote}
