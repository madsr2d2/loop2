\clearpage
\section{Production Facility Visit Notes}
\label{app:visitnotes}

This appendix summarizes the key insights and observations recorded during our group's visit to the Egholm production facility. The notes have been translated from the original Danish and reorganized for clarity.

\subsection{Optimization Potential}
    \begin{itemize}
    \item \textbf{Efficiency Gains:} There is significant potential for optimization during normal use. Some attachments can be improved by 10--15\%, while others show a potential improvement of 40--50\%.
    \item \textbf{New E-Tractor Goals:} The new electric tractor currently in development has a target runtime of a full 8-hour workday.
    \item \textbf{Cost Drivers:} The battery accounts for approximately 25\% of the total price of the E-Tractor. Reducing energy consumption would allow for a smaller, less expensive battery, directly improving the product's marketability.
\end{itemize}

\subsection{Product Design \& Interface}
\begin{itemize}
    \item \textbf{Complexity vs. Marketability:} The solution can be as complex as necessary, provided it remains profitable and marketable to the end-user.
    \item \textbf{User Interface Strategy:}
    \begin{itemize}
        \item \textit{Hard Functionality:} Critical controls like lifting/lowering tools or adjusting arms should remain on physical buttons and joysticks.
        \item \textit{Soft Functionality:} Parameters like cutting speed or suction power can be moved to touch screens or digital interfaces.
    \end{itemize}
\end{itemize}

\subsection{User Behavior \& Operation}
\begin{itemize}
    \item \textbf{User Mindset:} End-users typically do not prioritize fuel economy (e.g., km/L) since refuelling is easy. However, operational efficiency (e.g., driving slower to maintain performance while saving energy) is a key optimization avenue.
    \item \textbf{Current Habits:} Users often default to maximum power (``full throttle'') regardless of conditions. If the result is poor, they simply repeat the pass.
    \item \textbf{Lack of Guidance:} There is currently no training video or guide available to users on how to operate the equipment energy-efficiently.
\end{itemize}

\subsection{Technical Insights \& Questions}
\begin{itemize}
    \item \textbf{Context-Awareness:} Is the tool/motor needed right now? Could a ``start/stop'' function or an algorithm based on rain, speed, and grass length optimize the cutter blade speed?
    \item \textbf{Suction/Sweeping Optimization:}
    \begin{itemize}
        \item Does the suction unit need to run continuously?
        \item Alternative ideas included a ``dustpan'' mechanism or intermittent suction (collecting a pile and then shooting it into the hopper).
        \item The ``flow energy cost'' of the suction unit is constant whether it is actively collecting debris or not.
    \end{itemize}
    \item \textbf{Heating:} Heating is a major energy drain. Suggestions included distinguishing between summer/winter cabins or focusing heating on contact surfaces (steering wheel, seat) rather than the entire cabin air volume.
\end{itemize}

\subsection{Testing \& Validation}
\begin{itemize}
    \item \textbf{Current Testing:} Egholm performs tests, including cold-weather battery performance and snow tools in Switzerland.
    \item \textbf{Data Gaps:} There is currently insufficient data on optimal mowing speeds relative to grass length and moisture. More battery-based mowing tests are planned.
    \item \textbf{Focus:} The most important outcome for Egholm is a demonstrable solution that proves how energy is saved.
\end{itemize}

